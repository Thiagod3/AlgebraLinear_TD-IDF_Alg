\documentclass[12pt,a4paper]{article}
\usepackage[utf8]{inputenc}
\usepackage[T1]{fontenc}
\usepackage[brazil]{babel}
\usepackage{geometry}
\usepackage{graphicx}
\usepackage{amsmath}
\usepackage{booktabs}
\usepackage{float}
\usepackage{setspace}
\usepackage{hyperref}
\usepackage{tabularx}

\geometry{margin=2.5cm}
\setstretch{1.3}

\title{Análise de Similaridade de Reviews de Battlefield 6 utilizando TF-IDF}
\author{Thiago Vilela}
\date{Novembro de 2025}

\begin{document}

\maketitle

\section{Descrição do Dataset e Tema}

O dataset utilizado contém avaliações de usuários brasileiros sobre o jogo \textit{Battlefield 6} (BF6), coletadas da plataforma Steam. Cada linha representa uma review escrita por um jogador, contendo texto livre e a respectiva classificação (positiva ou negativa).

O tema escolhido foi a análise semântica de sentimentos negativos, com foco em identificar padrões linguísticos e possíveis ocorrências de ironia em comentários considerados negativos.

O objetivo principal foi verificar se, a partir de uma review modelo negativa, seria possível encontrar outras avaliações com conteúdo similar, mesmo que expressassem insatisfação de forma implícita ou irônica.

\section{Metodologia}

O processo de análise seguiu as seguintes etapas:

\begin{enumerate}
    \item \textbf{Leitura e pré-processamento do dataset:} Conversão dos textos para minúsculas, remoção de pontuação, números e caracteres especiais, além da exclusão de \textit{stopwords} da língua portuguesa.
    \item \textbf{Vetorização com TF-IDF (Term Frequency–Inverse Document Frequency):} Cada review foi transformada em um vetor numérico com base na frequência e relevância das palavras. Termos comuns a várias reviews receberam peso menor, enquanto palavras específicas (como \textit{lag}, \textit{rollback}, \textit{bug}) tiveram maior relevância.
    \item \textbf{Cálculo da Similaridade do Cosseno:} A similaridade do cosseno foi utilizada para comparar o vetor da review negativa modelo com todas as outras do dataset, gerando um valor entre 0 e 1 (sendo 1 a equivalência total).
    \item \textbf{Análise dos ângulos:} Como o cosseno reflete o ângulo entre os vetores, valores próximos de 1 indicam ângulos pequenos (alta similaridade semântica), enquanto valores menores indicam ângulos maiores (menor semelhança).
\end{enumerate}

\section{Resultados}

O comentário modelo selecionado foi o seguinte:

\begin{quote}
``Pedi devolução pois enfrentei diversos problemas de performance no modo multiplayer, com muito lag, rollback e input delay, mesmo tendo especificações acima do mínimo e com drivers atualizados... problemas que não rolaram no beta. Ainda vejo potencial no game, mas vou dar um tempo e esperar corrigirem esses problemas (que eu sei que nem todos estão enfrentando).''
\end{quote}

A similaridade calculada com as demais avaliações gerou os seguintes resultados (sem duplicatas):

\begin{table}[H]
\centering
\begin{tabularx}{\textwidth}{@{}cXl@{}}
\toprule
\textbf{Similaridade} & \textbf{Comentário resumido} & \textbf{Observação} \\ \midrule
1.000 & O próprio comentário modelo & --- \\
0.126 & ``O jogo é bom, mas tem vários problemas de rede, bugs e latência.'' & Vocabulário técnico similar \\
0.123 & ``Ainda não é o momento de comprar, há bugs e falhas no matchmaking.'' & Tom crítico construtivo \\
0.073--0.059 & Comentários positivos, destacando diversão ou gráficos & Vocabulário diferente \\
$\leq$ 0.05 & Reviews elogiando o jogo, sem defeitos & Sem similaridade textual \\ \bottomrule
\end{tabularx}
\caption{Resultados de similaridade entre a review modelo e outras avaliações.}
\end{table}

\section{Análise dos Ângulos}

Os valores de similaridade obtidos correspondem aos seguintes ângulos aproximados (em graus):

\begin{table}[H]
\centering
\begin{tabular}{@{}cc@{}}
\toprule
\textbf{Similaridade} & \textbf{Ângulo (°)} \\ \midrule
1.000 & 0 \\
0.126 & 82.8 \\
0.123 & 82.9 \\
0.073 & 85.8 \\
0.065 & 86.3 \\
0.059 & 86.6 \\
0.051 & 87.1 \\
0.040 & 87.7 \\ \bottomrule
\end{tabular}
\end{table}

Esses valores mostram que apenas dois textos apresentam ângulos relativamente menores (em torno de 83°), indicando maior proximidade semântica com o comentário modelo. As demais possuem ângulos mais abertos (acima de 85°), o que reflete diferença significativa de contexto e vocabulário.

\section{Discussão}

Apesar da review modelo apresentar críticas diretas à performance e rede, o modelo TF-IDF conseguiu identificar outros textos que expressam descontentamento técnico, ainda que de forma mais genérica.

Por outro lado, não foram encontradas reviews ironicamente negativas, pois o método TF-IDF analisa apenas a frequência e relevância de palavras, sem compreender o tom emocional ou o sarcasmo. Assim, expressões irônicas com vocabulário positivo (ex.: ``o jogo está incrível, pena que não funciona'') não seriam identificadas como similares.

\section{Conclusão}

O experimento demonstrou que o TF-IDF combinado com a Similaridade do Cosseno é eficaz para encontrar reviews com queixas semelhantes em termos técnicos e semânticos, mas apresenta limitações ao lidar com contextos irônicos ou ambíguos.

Para capturar esse tipo de sutileza, seria necessário empregar modelos de linguagem contextual, como BERT ou embeddings semânticos mais modernos.

Ainda assim, o método alcançou o objetivo proposto de identificar padrões de insatisfação em textos com estrutura e vocabulário próximos à review negativa inicial.

\end{document}
